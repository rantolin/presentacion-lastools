%==================================================================FRONT PAGE AND TOC
% For article only
\mode<presentation:0>{\thispagestyle{empty}\maketitle}

% For presentation only
\mode<presentation| article:0| handout:0>{
    \begin{frame}<article:0>[label=portada]
    \titlepage
    \end{frame}%Fin del frame
}

% For handout only
\mode<handout>{
  \begin{frame}[label=portada]
    \maketitle
  \end{frame}
}

% %% TABLE OF CONTENTS
% \begin{frame}[label=toc]
%     \mode<article:0>{\frametitle{Contents}}
%     \mode<presentation>{\small}
%     \tableofcontents[hidesubsections]
% \end{frame}

\begin{frame}[label=portada]
 \titlepage
\end{frame}
\begin{frame}[label=toc]
  \frametitle{Resumen}
\tableofcontents[hidesubsections]
\end{frame}
% \rowcolors{1}{ZurichBlue!20}{ZurichBlue!5}
%%==================================================================S INTRODUCTION
\section{LAStools}
%%==================================================================F 
\begin{frame}
  \frametitle{LAStools}
  \begin{enumerate}
    \item LASlib es una librería para la \alert{lectura} y \alert{escritura} de archivos en el
      estándar ASPRS LAS en C++
    \item Comandos para gestionar, manipular, transformar y procesar datos LiDAR
      en formato LAS
    \item \alert{LAStools}
      \begin{itemize}
        \item lasground.exe, lasheight.exe, lasclassify.exe, lasoverlap.exe,
          lascontrol.exe, lasgrid.exe, lastile.exe, lassort.exe, Lasclip.exe,
          lasinfo.exe, lasindex.exe, lasthin.exe, las2las.exe, lasboundary.exe,
          lasduplicate.exe, las2tin.exe,las2dem.exe, las2iso.exe, lasmerge.exe,
          lassplit.exe, lasprecision.exe, las2shp.exe, shp2las.exe, lasview.exe,
          laszip.exe, las2txt.exe, txt2las.exe
        \item las2las.cpp, las2txt.cpp, lasdiff.cpp, lasindex.cpp, lasinfo.cpp,
          lasmerge.cpp, lasprecision.cpp, laszip,cpp, txt2las.cpp
      \end{itemize}
  \end{enumerate}
\end{frame}
%%==================================================================F 
\begin{frame}
  \frametitle{Licencia}
  \begin{enumerate}
    \item Parte libre y abierta
      \begin{itemize}
        \item LASlib (con LASzip)
        \item herramientas principales: las2las, las2txt, laszip,\ldots
        \item Licencia \alert{LGPL}
      \end{itemize}
    \item Parte privativa y cerrada
      \begin{itemize}
        \item No es libre excepto para fines académicos o educacionales (con
          límites)
        \item La versión completa se puede licenciar
      \end{itemize}
  \end{enumerate}
\end{frame}
%%==================================================================F 
\begin{frame}
  \frametitle{Historia}
  \begin{enumerate}
    \item Inicio del desarrollo en Enero de 2007
      \begin{itemize}
        \item API para lectura/escritura de LAS
        \item lasinfo, lasview, last2txt, txt2las, laszip, las2las
      \end{itemize}
    \item Publica desde Abril de 2007
    \item Aparece libLAS como un \emph{fork} en Diciembre de 2007
    \item Comercializada desde 2010
      \begin{itemize}
        \item GUI + multi-procesador en 2011
        \item ArcGIS toolbox desde Abril de 2012
      \end{itemize}
    \item En internet
      \begin{itemize}
        \item \beamergotobutton{\url{http://groups.google.com/group/lastools}}
        \item \beamergotobutton{\url{http://facebook.com/lastools}}
        \item \beamergotobutton{\url{http://twitter.com/lastools}}
        \item \beamergotobutton{\url{http://www.linkedin.com/groups?gid=4408378}}
      \end{itemize}
  \end{enumerate}
\end{frame}
%%==================================================================F 
  \defverbatim[colored]\laszip{
     \begin{lstlisting}[style=shell]
        $ laszip lidar.las lidar.laz
        $ laszip lidar.laz lidar_copy.las
    \end{lstlisting}
  }
\begin{frame}
  \frametitle{LASzip}
  \begin{enumerate}
    \item Compresión de archivos .LAS sin \alert{pérdida}
    \item 7\% - 20\% del tamaño del archivo original
    \item Ganador del premio Geospatial World Forum 2012 Technology 
      Innovation Award para el procesado de datos LiDAR
    \item Incorporado en: LAStools, Global Mapper, Opals (TU Wien)...
    \item Utilizado por: NOAA, USGS, Fugro, Blom, Riegl, Dielmo...
    \laszip
  \end{enumerate}
\end{frame}
%%==================================================================S 
\section{Procesamiento con LAStools}
%%==================================================================Sb
\subsection{Control de Calidad}
%%==================================================================F
  \defverbatim[colored]\calidad{
    \begin{lstlisting}[language=bash,style=shell]
C:\> # Resumen de todo el contenido de los archvios LAS
C:\> lasinfo -i *.las -compute_density
C:\> # Inspeccion visual de los LAS
C:\> lasview -i *.las 
C:\> # Calcular el contorno y los huecos del vuelo LiDAR
C:\> lasboundary  -i *.las  -holes  -disjoint  -oshp
C:\> # Crear una malla con la densidad puntual y visualizarla en falso color
C:\> lasgrid -i *.las  -density  -step 3  -set_minmax 0 20 -false  -opng  -utm 28N
C:\> # Determinar si existen puntos repetidos
C:\> lasduplicate  -i *.las  -unique_xyz  -onil
C:\> # Comprobar la alineacion vertical y horizontal de las pasadas
C:\> lasoverlap  -i *.las  -step 3
    \end{lstlisting}
  }
\begin{frame}
  \frametitle{Control de Calidad}
\calidad
\end{frame}
%%==================================================================F
  \defverbatim[colored]\preparacion{
    \begin{lstlisting}[language=bash,style=shell]
C:\> # Mejora de los datos y reproyeccion
C:\> las2las  -i *.las -rescale 0.01 0.01 0.01  -utm 28N  -olaz  -odix  l
C:\> # Unir todos los archivos y dividir los datos en teselas
C:\> lastile -i *l.laz  -tile_size 500  -buffer 30  -olaz  -o tiles
C:\> # Clasificar puntos terreno (class 2)
C:\> lasground  -i tiles*.laz  -fine  -olaz  -odix g
C:\> # Calcular la altura de los objetos respecto al terreno
C:\> lasheight  -i tiles*g.laz  -olaz  -odix h
C:\> # Clasificar los puntos no-terreno en edificios
C:\> # (class 6) y vegetacion (class 5)
C:\> lasclassify  -i tiles*gh.laz  -olaz  -odix c
C:\> # Calcular la altura de los objetos y 
C:\> # reemplazar la anterior
C:\> lasheight  -i tiles*ghc.laz  -replace_z  -olaz  -odix h
    \end{lstlisting}
  }
\begin{frame}
  \frametitle{Preparación de los datos}
\preparacion
\end{frame}
%%==================================================================F
  \defverbatim[colored]\derivados{
    \begin{lstlisting}[language=bash,style=shell]
C:\> # Triangular puntos en un TIN y un raster (DSM)
C:\> las2dem  -i tiles*.laz  -first_only  -step 2.5  -use_tile_bb  -otif
C:\> # Triangular puntos en un TIN y un raster (DTM)
C:\> las2dem  -i tiles*g.laz  -keep_class 2  -step 2.5  -use_tile_bb  -ocut 1  -otif
C:\> # Encontrar la altura maxima en cada celda y exportar a PNG
C:\> lasgrid  -i tiles*ghch.laz  -step 2.5  -use_tile_bb  -highest -false  -ocut 4  -opng
C:\> # Estimar la densidad de la vegetacion baja, media y alta
C:\> # contando los puntos por celda que caen en diferentes 
C:\> # intervalos de altura: 30cm-99cm; 1m-1.99m; 2m-3.99m
C:\> lasgrid  -i tiles*ghch.laz  -step 2.5  -clip_z 0.3 0.99  -density  -odix low  -oasc
C:\> lasgrid  -i tiles*ghch.laz  -step 2.5  -clip_z 1.0 1.99  -density  -odix mid1  -oasc
C:\> lasgrid  -i tiles*ghch.laz  -step 2.5  -clip_z 2.0 3.99  -density  -odix mid2  -oasc
    \end{lstlisting}
  }
\begin{frame}
  \frametitle{Cartografía derivada: Estudios Forestales}
\derivados
\end{frame}
%%==================================================================F
  \defverbatim[colored]\derivados{
    \begin{lstlisting}[language=bash,style=shell]
C:\> # Generar un DTM sin errores de borde
C:\> # Crear una carpeta llamada 'lastools/bin/tiles'
C:\> las2dem -i tiles_classified\fitch*.laz -keep_class 2 -thin_with_grid 0.5 -extra_pass -use_tile_bb -odir tiles_dtm -obil -cores 4
C:\> # Generar un DSM sin errores de borde
C:\> las2dem -i tiles_classified\fitch*.laz -first_only -thin_with_grid 0.5 -extra_pass -use_tile_bb -odir tiles_dsm -obil -cores 4
C:\> # Visualizar lo datos
C:\> lasview -i tiles_dsm\fitch*.bil -gui
C:\> # Crear el mosaico 
C:\> lasgrid -i tiles_dsm\fitch*.bil -merge -utm 19N -vertical_navd88 -o dsm.tif
C:\> # Crear el mosaico de manera directa
C:\> blast2dem -i tiles_dtm\fitch*.bil -merged -hillshade -utm 19N -vertical_navd88 -o dtm_hillshade_raster.png
C:\> # Crear un mapa de curvas de nivel de manera directa
C:\> blast2iso -i tiles_dtm\fitch*.bil -merged -iso_every 1 -simplify_length 0.5 -simplify_area 1 -clean 5 -utm 19N -o dtm_contours_raster_1m.shp
C:\> # Delinear edificios
C:\> lasboundary -i tiles_final\fitch*.laz -merged -keep_class 6 -concavity 1.5 -disjoint -o buildings.shp
C:\> 
    \end{lstlisting}
  }
\begin{frame}
  \frametitle{Cartografía derivada: Estudios Forestales}
\derivados
\end{frame}

%%==================================================================F
\againframe{portada}
%%==================================================================F

%\section{Resumen}
